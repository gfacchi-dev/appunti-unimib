\mainmatter
\chapter{Introduzione}
Garzone è una soluzione informatica rivolta alla pubblica amministrazione, in particolare ad enti comunali, al servizio dei cittadini e del commercio locale. L'idea è nata dalla necessità di rilancio del commercio locale, oppresso dal progresso dei giganti tecnologici e dalla pandemia. Affiliata all'iniziativa "Soldiarietà Digitale" promossa dal Ministero dell’Innovazione Tecnologica\cite{MISE}, successivamente si è poi evoluta sulla base dei consigli di varie attività commerciali e di amministrazioni pubbliche. Il progetto, dopo la sua pubblicazione, ha inoltre ricevuto il "Premio Top of the Pid - Restart"\cite{PID}, promosso dalle Camere di Commercio italiane, come miglior modello di business 4.0 per il rilancio del commercio locale. L’iniziativa premia i migliori progetti innovativi che possono agevolare il rilancio dell’economia per uscire dalla profonda crisi provocata della diffusione del Covid-19.
\begin{figure}[!htb]
    \centering
    \includegraphics[width=0.3\textwidth]{pid.png}
    \caption{Premio Top of the Pid - Restart}
\end{figure}
\section{Specifiche progettuali}
\begin{figure}[!htb]
    \centering
    \includegraphics[width=0.4\textwidth]{garzone-logo.png}
    \caption{Logo di Garzone}
\end{figure}
La finalità principale di Garzone riguarda il soddisfacimento della necessità dei commercianti di poter aggiornarsi, tramite un primo approccio, al commercio online. La soluzione, per raggiungere l'obiettivo, deve permettere a un titolare di poter gestire con semplicità il proprio catalogo di prodotti venduti e servizi offerti e poterlo condividere online con la cittadinanza. Inoltre, in un contesto dove la presenza online è strettamente necessaria per far conoscere al pubblico la propria attività, un titolare deve avere a disposizione tutti gli strumenti per venire contattato e per far sapere dove svolge la sua attività tramite la costituzione di una vetrina virtuale. Fondamentale è poi l'interazione tra cittadino e commerciante, la quale deve essere garantita mediante l'implementazione di un'interfaccia per scambiarsi messaggi di chat e per una corretta gestione degli ordini di ritiro e di consegna a domicilio, eventualmente con pagamento online. \\Di enorme rilevanza, per avere una buona accessibilità alla piattaforma da parte dell'utenza, è inoltre previsto l'utilizzo di linee guida grafiche ispirate agli standard pubblicati da AgID (Agenzia per l'Italia digitale)\cite{AGID}. Gli enti comunali coinvolti dovranno invece avere a disposizione degli strumenti di controllo, per verificare il funzionamento dell'infrastruttura e il suo corretto utilizzo da parte dei commercianti. \\Inoltre sarà per loro possibile comunicare con l'utenza mediante l'apposita sezione dell'applicativo a loro dedicato. Infine dovrà essere prevista, tramite l'implementazione dedicata di una piattaforma, l'amministrazione di tutta l'infrastruttura inizialmente da parte dell'azienda, ma in futuro da terzi. 
\section{Organizzazione del lavoro}
Inizialmente lo stage prevedeva la presenza presso la sede aziendale, dove il coordinamento avveniva mediante riunioni giornaliere prefissate tra il team di sviluppo ed il tutor aziendale, generalmente il mattino e a fine giornata. Nel caso emergessero problematiche, causate in genere dall'inesperienza, risultava molto semplice interfacciarsi tra colleghi presenti in loco. Con il riaggravarsi della pandemia il lavoro è stato riorganizzato in modalità full-remote, tramite l'utilizzo della piattaforma Clickup. Le riunioni venivano svolte giornalmente il mattino e prevedevano, in aggiunta, un momento dove confrontarsi sui problemi emersi durante l'orario di lavoro. Dopo l'inserimento del collega Agazzi, il tutor aziendale ha previsto una suddivisione del lavoro tramite la definizione di task precisi, eseguibili solo dopo una formazione collettiva da lui fornita. A fine giornata era prevista una chiamata di allineamento tra il team di sviluppo per aggiornare l'un l'altro sugli sviluppi effettuati durante la giornata e per chiarire problematiche non risolte singolarmente. Il testing delle implementazioni veniva eseguito dal tutor aziendale, il quale si occupava di segnalare eventuali problematiche e possibili correzioni implementabili.
