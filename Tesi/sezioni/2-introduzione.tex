\mainmatter
\chapter{Introduzione}
Garzone è una soluzione informatica rivolta alla pubblica amministrazione, in particolare ad enti comunali, al servizio dei cittadini e del commercio locale. L'idea è nata dalla necessità di rilancio del commercio locale, oppresso dal progresso dei giganti tecnologici e dalla pandemia. Affiliata all'iniziativa "Soldiarietà Digitale" promossa dal Ministero dell’Innovazione Tecnologica, successivamente si è poi evoluta sulla base dei consigli di varie attività commerciali e di amministrazioni pubbliche. Il progetto, dopo la sua pubblicazione, ha inoltre ricevuto il "Premio Top of the Pid - Restart", promosso dalle Camere di Commercio italiane, come miglior modello di business 4.0 per il rilancio del commercio locale. L’iniziativa premia i migliori progetti innovativi che possono agevolare il rilancio dell’economia per uscire dalla profonda crisi provocata della diffusione del Covid-19.
\begin{figure}[!htb]
    \centering
    \includegraphics[width=0.3\textwidth]{pid.png}
    \caption{Premio Top of the Pid - Restart}
\end{figure}
\section{Riepilogo esperienza di stage}
Nel mese di Luglio 2020 vengo contattato dall'azienda dglen srl presso Milano. Dopo un colloquio conoscitivo e avermi introdotto le mansioni che mi sarebbero state assegnate, già dalla settimana successiva comincia il mio periodo di formazione. Il percorso formativo ha previsto inizialmente l'acquisizione di competenze sulla libreria Javascript "React" mediante l'utilizzo della piattaforma Udemy, una repository online di corsi di formazione. Successivamente mi sono state illustrate le modalità di lavoro da adottare durante lo sviluppo del nuovo progetto, come l'utilizzo di Bitbucket per condivisione del codice sorgente e di Visual Studio Code per la sua stesura. Nei mesi successivi le uniche mansioni assegnatemi riguardavano lo sviluppo e la manutenzione di Garzone. L'aspetto più significativo dell'esperienza di stage trascorsa è sicuramente rappresentato dal lavoro di gruppo, senza il quale non sarebbe stato per me possibile superare ostacoli e imparare tutte le metodologie di lavoro con costanza e meticolosità.

\section{Specifiche progettuali}
\begin{figure}[!htb]
    \centering
    \includegraphics[width=0.4\textwidth]{garzone-logo.png}
    \caption{Logo di Garzone}
\end{figure}
La finalità principale di Garzone riguarda il soddisfacimento della necessità dei commercianti di poter aggiornarsi, tramite un primo approccio, al commercio online. La soluzione, per raggiungere l'obiettivo, deve permettere a un titolare di poter gestire con semplicità il proprio catalogo di prodotti venduti e servizi offerti e poterlo condividere con efficacia online con la cittadinanza. Inoltre, in un contesto dove la presenza online è strettamente necessaria per far conoscere al pubblico la propria attività, un titolare deve avere a disposizione tutti gli strumenti per far sapere come venire contattato e dove svolge la sua attività tramite la costruzione di una vetrina virtuale. Fondamentale è poi l'interazione tra cittadino e commerciante, la quale deve essere garantita mediante l'implementazione di un'interfaccia per scambiarsi messaggi di chat e la corretta gestione degli ordini di ritiro e di consegna a domicilio, eventualmente con pagamento online. Di enorme rilevanza, per avere una buona accessibilità alla piattaforma da parte dell'utenza, è previsto l'utilizzo di linee guida grafiche ispirate agli standard pubblicati da AgID (Agenzia per l'Italia digitale). Gli enti comunali coinvolti dovranno invece avere a disposizione degli strumenti di controllo, per verificare il funzionamento dell'infrastruttura e il suo corretto utilizzo da parte dei commercianti. Inoltre sarà per loro possibile comunicare con l'utenza mediante l'apposita sezione dell'applicativo a loro dedicato. Infine dovrà essere prevista, tramite l'implentazione dedicata di una piattaforma, l'amministrazione di tutta l'infrastruttura inizialmente da parte dell'azienda, ma in futuro da terzi. 
